\documentclass[a4paper,10pt]{article}
\usepackage[brazilian]{babel}
\usepackage[T1]{fontenc}
\usepackage[utf8]{inputenc}
\usepackage{lmodern}
\usepackage{geometry}
\geometry{verbose,tmargin=3cm,bmargin=3cm,lmargin=2cm,rmargin=2cm}
\usepackage{fancyhdr}
\pagestyle{fancy}
\usepackage{amsthm}
\usepackage{amsmath}
%\numberwithin{equation}{section}
\usepackage{amsfonts}
\usepackage{amssymb}
\usepackage{mathtools}
%\usepackage{mnsymbol}
\usepackage[authoryear]{natbib}
\usepackage{algorithm}
\usepackage{algpseudocode}
\usepackage[hyphens]{url}
\usepackage{color}
\usepackage{graphicx}
\usepackage{xcolor}
\usepackage{import}
\usepackage{bm}
\usepackage{cancel}
\usepackage{listings}
\usepackage{graphicx}
\usepackage{caption}
\usepackage{subcaption}
\graphicspath{{Figures/}}

\usepackage{xcolor}

\definecolor{codegreen}{rgb}{0,0.6,0}
\definecolor{codegray}{rgb}{0.5,0.5,0.5}
\definecolor{codepurple}{rgb}{0.58,0,0.82}
\definecolor{backcolour}{rgb}{0.95,0.95,0.92}

\lstdefinestyle{mystyle}{
	backgroundcolor=\color{backcolour},   
	commentstyle=\color{codegreen},
	keywordstyle=\color{magenta},
	numberstyle=\tiny\color{codegray},
	stringstyle=\color{codepurple},
	basicstyle=\ttfamily\footnotesize,
	breakatwhitespace=false,         
	breaklines=true,                 
	captionpos=b,                    
	keepspaces=true,                 
	numbers=left,                    
	numbersep=5pt,                  
	showspaces=false,                
	showstringspaces=false,
	showtabs=false,                  
	tabsize=2
}

\lstset{style=mystyle}

\makeatletter
\usepackage[colorlinks=true,linkcolor=red]{hyperref}
\makeatother

\bibliographystyle{apalike2}

\begin{document}
% ------------------ NOTA -----------------------
%
% 1 - Você não precisa alterar o arquivo de template, somente esse arquivo.
% 2 - Abaixo de cada comando há uma nota que o auxiliará a preencher os campos. Todos os
%     comandos são auto-explicáveis, mas as notas o ajudarão em caso de dúvidas.
%
% -----------------------------------------------

\newcommand{\surname}{RAFFO}
% Substitua "SOBRENOME" por ADORNO, RAFFO ou PIMENTA, conforme quem for o seu orientador.
% No caso de haver um co-orientador, use as três primeiras letras de cada sobrenome separadas
% por '/'. Por exemplo, se seu orientador é o Prof. Adorno e seu co-orientador é o Prof.
% Pimenta, escreva ADO/PIM.

\newcommand{\initials}{JMC}
% Coloque suas iniciais aqui (ex., Bruno Vilhena Adorno = BVA)

\newcommand{\reportnumber}{1}
% Mude "X" pelo número do relatório

\newcommand{\reportversion}{1}
% Mude "Y" pelo número da versão do relatório.

\newcommand{\reporttitle}{\textbf{Parallel Distributed Control(PDC) de um Pendulo Invertido Via Representação Takagi-Sugeno}}

\newcommand{\registrationnumber}{2016086496}

\newcommand{\studentname}{Jonatan Mota Campos}

\newcommand{\advisorname}{Guilherme Vianna Raffo}

\newcommand{\coadvisorname}{}

\include{report_template/ReportTemplate-PT}

% % % % % % % % % % % % % % % % % % % % % % % % % % %
\newpage
\tableofcontents
\thispagestyle{empty}

\newpage
\pagenumbering{arabic}
% % % % % % % % % % % % % % % % % % % % % % % % % % %
\newcommand{\zero}{\bm{\emptyset}}

\section{Modelagem Dinâmica}
\subsection{Formulação Newton Euler e Representação Via Sistema Descritor}
\subsection{Representação Takagi-Sugeno}
\section{Controle}
\subsection{Lei de Controle PDC}
\subsection{Lei de Controle PDC com $\mathcal{H}_\infty$}



\end{document}
